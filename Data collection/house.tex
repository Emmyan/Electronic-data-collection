\documentclass[]{report}


% Title Page
\title{\begin{Huge}
		HOUSEHOLD SURVEY.
	\end{Huge}}
\author{OCAKACON EMMY KILAMA}


\begin{document}
\maketitle

\begin{abstract}
	\section{Introduction to the study.}
	Today in Uganda, most people are experiencing the effects of socio-economic problems in most of the families. Due to the high cost of living in mostly urban areas, most of the population have resorted to living in slums where they feel live is a little bit easier. The government needs to identify what causes the hardships of live in the urban, and also how families are coping with the live and find solutions to this.
	
	\section{Statement of the Problem.}
	The purpose of the study was to find out about the socio-economic status of families because it would help in economic planning by the government.
	
	\section{Significance of the study.}
	There are two groups that would benefit from this study. First of all it is the government since knowing the socio-economic status of the citizens helps in effective resource allocation, that is, proper planning. Then the citizens themselves also benefit due the proper planning by the government.
	
	\section{Scope of the study.}
	The study was conducted with each individual in each of the households visited without skipping any unless they were babies. It was also restricted to specific districts of Uganda. For the purpose of this study, socio-economic is defined as status involving both social and economic factors.
	
	\section{Method of the study.}
	Source of data.
	Data for the study was collected using questionnaire developed by Ocakacon Emmy Kilama. The questionnaire consisted of four sections. First section identifies where a respondent is from, second section talks about who is in the family, third about economic status of a family and the fourth about kind of construction the live in.
	Sample Methods.
	The respondents involved in this survey were individual members of a family from the selected districts of Uganda.
	
	\section{Limitations of the study.}
	Limitations of the study.
	This study was limited through the use of questionnaire as a data collection instrument. Because questionnaires must be brief areas that may be affected by socio-economic problems may have not all been included in the questionnaires.
	
	\section{Conclusion.}
	On the basis of this study, several conclusions concerning effects of socio-economic status of families can be drawn. The findings of this study indicates that poverty is very rampant in the different parts of the country. Health and other living conditions were found to been the worst effect of status of families. Therefore the government needs to come up with effective plans for eradicating poverty in the country.
	
	
	\section{Recommendations.}
	Based on the findings of this study, the following recommendations are made:
	
    \begin{itemize}
    	\item 	The government should offer soft loans so that people can find ways of engaging in different economic activities.
    	\item 	The government should also help to provide public facilities for those who cannot afford on their own.
    	\item 	People should also stop over depending on the government to solve their needs. 
    \end{itemize}
	
\end{abstract}

\end{document}          
